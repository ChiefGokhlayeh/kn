% !TeX root = ../main.tex
\chapter{\glsentrylong{pke}}

Gerade moderne Luxusfahrzeuge werben mit dem zusätzlichen Komfort von \glsreset{pke}\gls{pke}. Oft wird dies auch auf die elektronische Wegfahrsperre ausgeweitet und kann dann als \glsreset{pkes}\gls{pkes} bezeichnet werden.

In diesem Kapitel wird die Funktionsweise solcher Systeme beschrieben. Zunächst soll grundlegende auf das in der Praxis weit verbreitete Verfahren mittels \foreignlanguage{english}{Challenge \& Response Handshake} eingegangen werden. Dabei wird explizit die Anfälligkeit für \foreignlanguage{english}{Replay}-Attacken herausgearbeitet. Es wird außerdem ein in der Literatur seit längerem bekanntes Verfahren zur Absicherung gegen solche Angriffe vorgestellt. Die hergeleitete Theorie mit den Bedrohungen der realen Praxis verglichen. Als Beispiel dient hier das DST40 Verfahren, entwickelt von Texas Instruments und unter anderem verwendet in Luxusmodellen wie dem Tesla Model S~\cite{Wouters2019}.

\section{Funktionsprinzip}

In einem \gls{pke} System geschieht die Kommunikation mit dem Fahrzeug \emph{passiv}, d.\,h., alleinig \emph{die Präsenz} eines Schlüssels in der unmittelbaren Umgebung zum Fahrzeug reicht aus, um dieses zu entsperren. Anders als bei herkömmlichen \gls{rke} Systemen ist jenseits des Tragens eines Schlüssels keine weitere Interaktion des Nutzers notwendig. In der Praxis wird dazu i.\,d.\,R.\@ ein bidirektionaler Kommunikationskanal zwischen Fahrzeug und Schlüssel aufgebaut. Um die Batterielebenszeit des Schlüssels zu verlängern, wird die Funkkommunikation vom Fahrzeug aus initiiert. Der Schlüssel horcht dabei zunächst auf die Präsenz eines \gls{lw}[n] Signals vom Fahrzeug bevor er selbst aktiv wird. Detektiert ein Schlüssel das Signal eines Fahrzeugs antwortet er über einen \gls{ukw}[n] Funkkanal~\cite{Alrabady2005,Verdult2015,Garcia2016}.

\subsection{\foreignlanguage{english}{Challenge \& Response}}

\subsection{\foreignlanguage{english}{Distance Bounding}}

\section{Bedrohungen}
