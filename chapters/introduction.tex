% !TeX root = ../main.tex
\chapter{Einführung}

\foreignlanguage{english}{Keyless-Entry} Systeme ermöglichen das kontaktlose Entsperren von Fahrzeugen ohne mechanischen Schlüssel. Im Automobilbereich seit langer Zeit verbreitet sind \foreignlanguage{english}{Remote Keyless-Entry} (RKE) und \foreignlanguage{english}{Passive Keyless-Entry and Start} (PKES) Systeme. In beiden Ansätzen wird die Authentifizierung meist über Funk durchgeführt, wobei bei vielen neueren Modellen bereits vollständig auf einen mechanischen Schlüssel verzichtet wird. Einige ältere Systeme setzen auf einen Datenaustausch über Infrarot~\cite[S.\ 930]{Garcia2016}. Auf diese wird im Weiteren jedoch aufgrund der schwindenden Verbreitung nicht weiter eingegangen.

Bei RKE initiiert der Nutzer die Authentifizierung aktiv, z.B. per Knopfdruck auf dem Fahrzeugschlüssel. Das Funkmodul im Schlüssel sendet daraufhin eine authentifizierte Anweisung an das entsprechende Steuergerät im Fahrzeug. Nur bei erfolgreicher Authentifizierung schaltet das Auto die Türen frei. PKES hingegen erfordert keine Involvierung des Nutzers. Die Fahrzeugtüren werden bei Annäherung des Schlüssels an das Fahrzeug und erfolgreicher Authentifizierung automatisch freigeschaltet. Neben dem Zugang zum Fahrzeug kann auch der Motorstart an die Präsenz eines Schlüssels gekoppelt werden. Nur, wenn sich der Schlüssel innerhalb des Fahrzeugs befindet erlaubt die elektronische Wegfahrsperre den Start des Motors. Zur Umsetzung dieser Funktion kommen RFID und NFC Techniken zur Anwendung~\cite{Rainer2010}, die separat zu den Techniken von \foreignlanguage{english}{Keyless-Entry} zu betrachten sind.

Diese Seminararbeit befasst sich mit der Kommunikationssicherheit von RKE und PKES Systemen im Automotive-Bereich, wobei die hierbei verwendeten Verfahren auch für nicht-Automotive-Anwendungen geeignet sind. Es soll zunächst auf klassische Verfahren zur elektronischen Zugriffskontrolle bei Automobilen eingegangen werden. Darunter fallen speziell die RKE Systeme, die per Knopfdruck auf den Schlüssel aktiviert werden. Dadurch soll ein grundlegendes Verständnis über die Bedrohungen, denen solche Systeme ausgesetzt sind, gebildet werden. Besonderes Augenmerk wird auf die Diskussion von \foreignlanguage{english}{Relay}-Attacken und deren Abwehr gelegt. Hierbei kommen unter anderem sogenannte \foreignlanguage{english}{Distance-Bounding} Protokolle zum Einsatz, deren Funktionsweise anhand eines Beispiels eingängig erklärt wird.
