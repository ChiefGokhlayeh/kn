% !TeX root = ../main.tex
\chapter{\glsentrylong{pke}}

Gerade moderne Luxusfahrzeuge werben mit dem zusätzlichen Komfort von \glsreset{pke}\gls{pke}. Oft wird dies auch auf die elektronische Wegfahrsperre ausgeweitet und kann dann als \glsreset{pkes}\gls{pkes} bezeichnet werden.

In diesem Kapitel wird die Funktionsweise solcher Systeme beschrieben. Zunächst soll grundlegende auf das in der Praxis weit verbreitete Verfahren mittels \foreignlanguage{english}{Challenge \& Response Handshake} eingegangen werden. Dabei wird explizit die Anfälligkeit für \foreignlanguage{english}{Replay}-Attacken herausgearbeitet. Es wird außerdem ein in der Literatur seit längerem bekanntes Verfahren zur Absicherung gegen solche Angriffe vorgestellt. Die hergeleitete Theorie mit den Bedrohungen der realen Praxis verglichen. Als Beispiel dient hier das DST40 Verfahren, entwickelt von Texas Instruments und unter anderem verwendet in Luxusmodellen wie dem Tesla Model S~\cite{Wouters2019}.

\section{Funktionsprinzip}

\subsection{Challenge \& Response}

\subsection{\foreignlanguage{english}{Distance Bounding}}

\section{Bedrohungen}
