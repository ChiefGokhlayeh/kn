\documentclass[
  draft=false,
  paper=a4,
  twoside=false,
  fontsize=11pt,
  headsepline,
  BCOR=10mm,
  DIV=11
]{scrbook}
\usepackage[ngerman,english]{babel}
%% see http://www.tex.ac.uk/cgi-bin/texfaq2html?label=uselmfonts
\usepackage[T1]{fontenc}
\usepackage[utf8]{inputenc}
\usepackage{csquotes}
\usepackage{libertine}
\usepackage{pifont}
\usepackage{microtype}
\usepackage{booktabs}
\usepackage{subcaption}
\usepackage{textcomp}
\usepackage[german,refpage]{nomencl}
\usepackage{setspace}
\usepackage[locale=DE]{siunitx}
\usepackage{makeidx}
\usepackage{listings}
\usepackage[style=iso-numeric,minnames=1,maxnames=3,giveninits,uniquename=init]{biblatex}
\usepackage{amsmath}
\usepackage[section]{placeins}
\usepackage{amssymb}
\usepackage[ngerman,colorlinks=true]{hyperref}
\usepackage[ngerman]{cleveref}
\usepackage{nameref}
\usepackage{menukeys}
\usepackage{soul}
\usepackage{hawstyle}
\usepackage{lipsum} %% for sample text
\usepackage{scrhack}
\usepackage{xspace}
\usepackage{tikz}

%% add bib files
\addbibresource{references.bib}

%% load tikz libraries
\usetikzlibrary{
  arrows.meta,
  calc,
  chains,
  matrix,
  positioning,
  shapes.geometric,
}

%% set menukeys style
\renewmenumacro{\directory}[>]{pathswithblackfolder}

%% define custom commands
\newcommand{\defectcrack}{\texttt{crack}\xspace}
\newcommand{\defectdarkarea}{\texttt{darkarea}\xspace}
\newcommand{\defectfinger}{\texttt{finger}\xspace}
\newcommand{\defectok}{\texttt{ok}\xspace}

%% define some colors
\colorlet{BackgroundColor}{gray!20}
\colorlet{KeywordColor}{blue}
\colorlet{CommentColor}{black!60}
%% for tables
\colorlet{HeadColor}{gray!60}
\colorlet{Color1}{blue!10}
\colorlet{Color2}{white}

%% configure colors
\HAWifprinter{
  \colorlet{BackgroundColor}{gray!20}
  \colorlet{KeywordColor}{black}
  \colorlet{CommentColor}{gray}
  % for tables
  \colorlet{HeadColor}{gray!60}
  \colorlet{Color1}{gray!40}
  \colorlet{Color2}{white}
}{}

\lstset{%
  numbers=left,
  numberstyle=\tiny,
  stepnumber=1,
  numbersep=5pt,
  basicstyle=\ttfamily\small,
  keywordstyle=\color{KeywordColor}\bfseries,
  identifierstyle=\color{black},
  commentstyle=\color{CommentColor},
  backgroundcolor=\color{BackgroundColor},
  captionpos=b,
  fontadjust=true
}
\lstset{escapeinside={(*@}{@*)}, % used to enter latex code inside listings % chktex 9
        morekeywords={uint32_t, int32_t}
}
\Ifpdfoutput{
  \hypersetup{bookmarksopen=false,bookmarksnumbered,linktocpage}
}{}

%% more fancy C++
\DeclareRobustCommand{\cxx}{C\raisebox{0.25ex}{{\scriptsize +\kern-0.25ex +}}}

\clubpenalty=10000
\widowpenalty=10000
\displaywidowpenalty=10000

% unknown hyphenations
\hyphenation{
}

%% recalculate text area
\typearea[current]{last}

\makeindex
\makenomenclature{}

\begin{document}
\selectlanguage{ngerman}

%%%%%
%% customize (see readme.pdf for supported values)
\HAWThesisProperties{
  Author={Andreas Baulig},
  Title={Keyless-Entry Systeme im Automotive-Bereich},
  ThesisType={Seminararbeit},
  ThesisExperts={Dr.\@ Aining Li},
  ReleaseDate={19. Juni 2022}
}

%% title
\frontmatter

%% output title page
\maketitle

\newpage
\singlespacing{}

\tableofcontents
\newpage
%% enable if these lists should be shown on their own page
\listoftables
\listoffigures

%% main
\mainmatter{}
\onehalfspacing{}
%% write to the log/stdout
\typeout{===== File: chapter 1}
%% include chapter file (chapter1.tex)
%%\include{chapter1}
% !TeX root = ../main.tex
\chapter{Einführung}

\foreignlanguage{english}{Keyless-Entry} Systeme ermöglichen das kontaktlose Entsperren von Fahrzeugen ohne mechanischen Schlüssel. Im Automobilbereich seit langer Zeit verbreitet sind \gls{rke} und \gls{pke} Systeme. In beiden Ansätzen wird die Authentifizierung über einen Radiokanal durchgeführt, wobei bei vielen neueren Modellen bereits vollständig auf einen mechanischen Schlüssel verzichtet wird.

Bei \gls{rke} initiiert der Nutzer die Authentifizierung aktiv, z.\,B.\@ per Knopfdruck auf dem Fahrzeugschlüssel. Diese Systeme werden daher seltener auch als \gls{ake} bezeichnet. Das Funkmodul im Schlüssel sendet bei Aktivierung eine authentifizierte Anweisung an das entsprechende Steuergerät im Fahrzeug. Nur bei erfolgreicher Authentifizierung schaltet das Auto die Türen frei. \gls{pke} hingegen erfordert keine aktive Involvierung des Nutzers. Die Fahrzeugtüren werden bei Annäherung des Schlüssels an das Fahrzeug und erfolgreicher Authentifizierung automatisch freigeschaltet. Neben dem Zugang zum Fahrzeug kann auch der Motorstart an die Präsenz eines Schlüssels gekoppelt werden. Nur, wenn sich der Schlüssel innerhalb des Fahrzeugs befindet erlaubt die elektronische Wegfahrsperre den Start des Motors. Solche Systeme werden in der Literatur als \gls{pkes} bezeichnet. Zur Umsetzung dieser Funktion kommen \gls{rfid} und \gls{nfc} Techniken zur Anwendung~\cite{Rainer2010}, die separat zu den Techniken von \foreignlanguage{english}{Keyless-Entry} zu betrachten sind.

Diese Seminararbeit befasst sich mit der Kommunikationssicherheit von \gls{rke} und \gls{pke} Systemen im Automotive-Bereich, wobei die hierbei verwendeten Verfahren auch für nicht-Automotive-Anwendungen Gebrach finden. Es soll zunächst auf klassische Verfahren zur elektronischen Zugriffskontrolle bei Automobilen eingegangen werden. Darunter fallen speziell die \gls{rke} Systeme, die per Knopfdruck auf den Schlüssel aktiviert werden. Dadurch soll ein grundlegendes Verständnis über die Bedrohungen, denen solche Systeme ausgesetzt sind, gebildet werden. Besonderes Augenmerk wird auf die Diskussion von \foreignlanguage{english}{Relay}-Attacken und deren Abwehr gelegt. Hierbei kommen unter anderem sogenannte \foreignlanguage{english}{Distance-Bounding} Protokolle zum Einsatz, deren Funktionsweise im Rahmen der \gls{pke} Systeme eingängig erklärt wird.

% !TeX root = ../main.tex
\chapter{\glsentrylong{rke}}

Dieses Kapitel behandelt die Funktionsweise und potenzielle Angriffe auf \gls{rke} Systeme. Zunächst wird eine kurze Historie zu verschiedenen Funktionsansätzen gegeben, um anschließend näher auf das in aktuellen Fahrzeugen weit verbreitete Hitag2 einzugehen. Schließlich wird zu jedem erläuterten Verfahren eine Reihe von Angriffen vorgestellt.

\section{Funktionsprinzip}

Bei \gls{rke} Systemen sendet der Schlüssel bei Betätigung eines Tasters auf dem Gehäuse ein kodiertes Funksignal an das Fahrzeug. Die Reichweite beträgt je nach Modell einige \SIrange{10}{100}{\meter}. Für die Funkübertragung kommen für gewöhnlich die lizenzfreien \gls{ism} Bänder infrage (EU\@: \SIlist{433;868}{\mega\hertz}, USA\@: \SI{315}{\mega\hertz}).

Die einfachsten Varianten von \gls{rke} setzen auf unidirektionale Kommunikation mittels statischer Codes (sog.\@ \foreignlanguage{english}{Fixed Codes}). \Cref{fig:fixed_code_msc} zeigt das Nachrichtenablaufdiagramm dieses Verfahrens. Unter Verzicht auf jegliche kryptografische Sicherheit, sendet der Schlüssel bei Aktivierung hierbei eine feste Bit-Folge, die einer Funktion \(f \in {\{\mathrm{open}, \mathrm{close}\}}\) und Schlüsselkennung \(\mathit{id}\) zugehörig ist~\cite{Alrabady2005}. In~\cite{Garcia2016} konnten solche Systeme bei Modellen aus dem Baujahr 2000 ausfindig gemacht werden.

\begin{figure}
    \centering
    \begin{sequencediagram}
        \tikzstyle{inststyle}+=[text depth=0.35ex, inner sep=0.25cm]
        \newthread{user}{Nutzer}
        \newinst{key}{Schlüssel}
        \newinst{car}{Fahrzeug}

        \begin{call}{user}{betätige}{key}{}
            \begin{messcall}{key}{\({\{\mathit{id}, f\}}\)}{car}
                \begin{callself}{car}{prüfe Code}{}
                \end{callself}
                \begin{sdblock}{Wenn}{Code Korrekt?}
                    \begin{callself}{car}{öffne Türen}{}
                    \end{callself}
                \end{sdblock}
            \end{messcall}
        \end{call}
    \end{sequencediagram}
    \caption{Nachrichtenablaufdiagramm von \foreignlanguage{english}{Fixed Code} \glsentryshort{rke}.\label{fig:fixed_code_msc}}
\end{figure}

\begin{figure}
    \centering
    \begin{tikzpicture}[
            block/.style={
                    draw,
                    thick,
                    text depth=0.35ex,
                    inner sep=0.25cm,
                    font=\scriptsize,
                },
            extern label/.style={
                    font=\tiny,
                },
            node distance=0.4cm,
        ]
        \node [block] (rke_rf) {RKE RF};
        \node [bareantenna, left=of rke_rf.north west, yshift=0.2cm] (rke_antenna) {};
        \node [block, below=of rke_rf] (uc) {µC};
        \node [draw=gray, very thick, fill=black, circle, right=of uc] (button) {};
        \node [draw, fit=(button)] (button_case) {};
        \node [extern label, right=0 of button_case] (button_label) {Knopf};
        \node [block, below=of uc] (immo) {Wgfsp. RFID};
        \node [bareantenna, left=of immo.north west, yshift=0.2cm] (immo_antenna) {};

        \draw (immo_antenna) |- (immo);
        \draw [dotted] (immo) -- (uc) node [extern label, anchor=west, midway] {optional};
        \draw (button_case) -- (uc);
        \draw (rke_rf) -- (uc);
        \draw (rke_antenna) |- (rke_rf);

        \draw [decorate, decoration={expanding waves, angle=20, segment length=0.15cm}] ($(immo_antenna) - (0.35, 1)$) -- ($(immo_antenna) - (0.35, 2)$);
        \draw [decorate, decoration={expanding waves, angle=20, segment length=0.15cm}] ($(immo_antenna) - (-0.35, 2)$) -- ($(immo_antenna) - (-0.35, 1)$);
        \node [extern label, below=2 of immo_antenna] {Fahrzeug Wgfsp. (\SI{125}{\kilo\hertz})};

        \draw [decorate, decoration={expanding waves, angle=20, segment length=0.15cm}] ($(rke_antenna) + (0, 1)$) -- ($(rke_antenna) + (0, 2)$);
        \node [extern label, above=2 of rke_antenna.center] {Fahrzeug RKE (\SIlist[list-separator=/, list-final-separator=/]{433;315;868}{\mega\hertz})};

        \begin{scope}[on background layer]
            \node [fill=black!5, rounded corners=0.75cm, fit=(rke_rf)(rke_antenna)(uc)(button)(button_case)(button_label)(immo)(immo_antenna)] (key_housing) {};
            \path [fill=black!5] ($(key_housing.east) + (-0.1, 0.5)$) -- ++(5, 0) -- ++(0, -0.9) -- ++(-5, 0);
            \path [fill=black!12] ($(key_housing.east) + (4.9, 0.4)$) -- ++(-0.4, 0) -- ++(-0.2, -0.2) -- ++(-0.4, 0) -- ++(-0.1, 0.1) -- ++(-0.4, 0) -- ++(-0.1, 0.1) -- ++(-0.4, 0) -- ++(-0.2, -0.2) -- ++(-0.4, 0) -- ++(-0.2, 0.2) -- ++(-0.4, 0) -- ++(-0.2, -0.2) -- ++(-0.4, 0) -- ++(-0.2, 0.2) -- ++(-0.8, 0) -- ++(0, -0.45) -- ++(0.25, -0.25) -- ++(4.55, 0) node [midway, anchor=north, extern label] {mechanischer Schlüssel};
        \end{scope}
    \end{tikzpicture}
    \caption[Aufbau eines typischen \foreignlanguage{english}{rolling-code} \glsentryshort{rke} Schlüssels mit unidirektionaler Kommunikation.]{Aufbau eines typischen \foreignlanguage{english}{rolling-code} \glsentryshort{rke} Schlüssels mit unidirektionaler Kommunikation~\cite{Garcia2016}.}
\end{figure}

\section{Bedrohungen}

% !TeX root = ../main.tex
\chapter{\glsentrylong{pke}}

Gerade moderne Luxusfahrzeuge werben mit dem zusätzlichen Komfort von \glsreset{pke}\gls{pke}. Oft wird dies auch auf die elektronische Wegfahrsperre ausgeweitet und kann dann als \glsreset{pkes}\gls{pkes} bezeichnet werden.

In diesem Kapitel wird die Funktionsweise solcher Systeme beschrieben. Dabei wird die hergeleitete Theorie mit einer praktischen Implementierung eines \gls{pkes} Systems ergänzt. Als Beispiel dient hier das DST40 Verfahren, entwickelt von Texas Instruments und unter anderem verwendet in Luxusmodellen wie dem Tesla Model S.

\section{Funktionsprinzip}

\section{Bedrohungen}

% !TeX root = ../main.tex
\chapter{Zusammenfassung}

In dieser Seminararbeit werden die Funktionsweise und mögliche Angriffe auf moderne \gls{rke} und \gls{pke} Systeme erläutert. Es wird auf Kommunikationskanal und Nachrichtenaustausch zwischen Fahrzeug und Schlüssel eingegangen. Gezeigt wird, dass viele Systeme vulnerabel gegenüber \foreignlanguage{english}{Replay} oder \foreignlanguage{english}{Relay} sind und dass einige Hersteller in der Provisionierung ihrer eigentlich nach dem Stand der Technik ausgelegten Sicherheitssysteme versagt haben.

Auch wenn manche Hersteller denen Sicherheitslücken nachgewiesen werden konnten Nachbesserungen eingeleitet haben, so ist dennoch davon auszugehen, dass einige Millionen ältere Fahrzeuge immer noch angreifbar sind.

Gleichzeitig sollte nicht vergessen werden, welches Schutzmaß ein sicher ausgelegtes Schlüsselsystem maximal erreichen kann. Auch das sicherste System lässt sich durch Einschlagen der Fensterscheibe aushebeln. Es bleibt dennoch zu hoffen, dass Hersteller ihre Systeme nachbessern und --sollte dies aus Kostengründen nicht möglich sein-- die Option zum Deaktivieren der \gls{pke} Funktion erlauben.


%%%%
%% add some text to generate a sample document
%%%%

%% appendix if used
%%\appendix
%%\typeout{===== File: appendix}
%%\include{appendix}

% bibliography and other stuff
\backmatter{}

\typeout{===== Section: literature}
%% read the documentation for customizing the style
\printbibliography{}

% \typeout{===== Section: nomenclature}
%% uncomment if a TOC entry is needed
%%\addcontentsline{toc}{chapter}{Glossar}
% \renewcommand{\nomname}{Glossar}
% \clearpage
% \markboth{\nomname}{\nomname} %% see nomencl doc, page 9, section 4.1
% \printnomenclature

%% index
% \typeout{===== Section: index}
% \printindex

% \HAWasurency

\end{document}
