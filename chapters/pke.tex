% !TeX root = ../main.tex
\chapter{\glsentrylong{pke}}

Gerade moderne Luxusfahrzeuge werben mit dem zusätzlichen Komfort von \glsreset{pke}\gls{pke}. Oft wird dies auch auf die elektronische Wegfahrsperre ausgeweitet und kann dann als \glsreset{pkes}\gls{pkes} bezeichnet werden.

In diesem Kapitel wird die Funktionsweise solcher Systeme beschrieben. Zunächst soll grundlegende auf das in der Praxis weit verbreitete Verfahren mittels \foreignlanguage{english}{Challenge \& Response Handshake} eingegangen werden. Dabei wird explizit die Anfälligkeit für \foreignlanguage{english}{Replay}-Attacken herausgearbeitet. Es wird außerdem ein in der Literatur seit längerem bekanntes Verfahren zur Absicherung gegen solche Angriffe vorgestellt. Die hergeleitete Theorie mit den Bedrohungen der realen Praxis verglichen. Als Beispiel dient hier das DST40 Verfahren, entwickelt von Texas Instruments und unter anderem verwendet in Luxusmodellen wie dem Tesla Model S~\cite{Wouters2019}.

\section{Funktionsprinzip}

In einem \gls{pke} System geschieht die Kommunikation mit dem Fahrzeug \emph{passiv}, d.\,h., alleinig \emph{die Präsenz} eines Schlüssels in der unmittelbaren Umgebung zum Fahrzeug reicht aus, um dieses zu entsperren. Anders als bei herkömmlichen \gls{rke} Systemen ist jenseits des Tragens eines Schlüssels keine weitere Interaktion des Nutzers notwendig. In der Praxis wird dazu i.\,d.\,R.\@ ein bidirektionaler Kommunikationskanal zwischen Fahrzeug und Schlüssel aufgebaut. Um die Batterielebenszeit des Schlüssels zu verlängern, wird die Funkkommunikation vom Fahrzeug aus initiiert. Der Schlüssel horcht dabei zunächst auf die Präsenz eines \gls{lw}[n] Signals vom Fahrzeug bevor er selbst aktiv wird. Detektiert ein Schlüssel das Signal eines Fahrzeugs antwortet er über einen \gls{ukw}[n] Funkkanal~\cite{Alrabady2005,Verdult2015,Garcia2016}.

\subsection{\foreignlanguage{english}{Challenge \& Response}}

Im \foreignlanguage{english}{Challenge \& Response} Verfahren, gezeigt in \cref{fig:challenge_response_msc}, authentifiziert sich ein Schlüssel gegenüber einem Fahrzeug, indem es eine vom Fahrzeug vorgegebene \emph{Challenge} \(C\) (engl.\@ Herausforderung) beantwortet. Gemäß Kerckhoffs's Prinzip ist der Algorithmus \(h(C, K)\) zur Lösung der \foreignlanguage{english}{Challenge} auch einem potenziellen Angreifer bekannt. Daher verwalten beide Parteien einen gemeinsamen Geheimschlüssel \(K\), der in die Berechnung der \foreignlanguage{english}{Response} \(R_{\mathrm{car}|\mathrm{key}}\) (engl.\@ Antwort) einfließt. Die Berechnung von \(h(C, K)\) wird sowohl vom Schlüssel (dem \emph{Prüfling}) als auch dem Fahrzeug (dem \emph{Prüfer}) durchgeführt. Nur wenn beide Ergebnisse übereinstimmen, gilt die Prüfung als erfolgreich. Da in die \foreignlanguage{english}{Response} normalerweise keinerlei Zusatzinformationen kodiert sind, reicht es meistens aus, für \(h(C, K)\) eine kryptografische Einwegfunktion einzusetzen. Dennoch sind auch vermehrt kryptografische Block-Chiffren in der Praxis anzutreffen~\cite{Wouters2019}.

Da der Schlüssel bedingt durch dessen Formfaktor meist über eine limitierte Batterielaufzeit verfügt, befindet sich dieser üblicherweise in einem energiesparenden Schlafmodus. Um einen potenziell in der Umgebung befindlichen Schlüssel zu wecken, sendet das Fahrzeug periodisch Weckrufe. Sobald ein Weckruf empfangen wird, antwortet der Schlüssel mit seiner Schlüsselkennung \(\mathit{id}\). Wenn diese als zulässige Schlüsselkennung im Fahrzeug hinterlegt ist, initiiert das Fahrzeug eine Prüfung des Schlüssels mittels \foreignlanguage{english}{Challenge \& Response} Verfahren.

\begin{figure}
    \centering
    \begin{sequencediagram}
        \tikzstyle{inststyle}+=[text depth=0.35ex, inner sep=0.25cm]
        \newthread{car}{Fahrzeug}
        \tikzstyle{inststyle}+=[below right=-0.85cm and 5cm of car]
        \newinst{key}{Schlüssel}

        \mess{car}{Hallo?}{key}
        \postlevel{}
        \mess{car}{Hallo?}{key}
        \postlevel{}
        \begin{messcall}{car}{Hallo?}{key}
            \begin{messcall}{key}{\({\{\mathit{id}\}}\)}{car}
            \end{messcall}
        \end{messcall}

        \begin{callself}{car}{neue \foreignlanguage{english}{Challenge}}{\(C\)}
        \end{callself}

        \begin{messcall}{car}{\({\{C\}}\)}{key}
            \begin{callself}{key}{\foreignlanguage{english}{Response} \(h(C, K)\)}{\(R_{\mathrm{key}}\)}
            \end{callself}
            \begin{messcall}{key}{\({\{R_{\mathrm{key}}\}}\)}{car}
            \end{messcall}
        \end{messcall}
        \prelevel{}
        \prelevel{}
        \prelevel{}
        \prelevel{}
        \prelevel{}
        \begin{callself}{car}{\foreignlanguage{english}{Response} \(h(C, K)\)}{\(R_{\mathrm{car}}\)}
        \end{callself}

        \postlevel{}
        \postlevel{}
        \postlevel{}
        \begin{sdblock}{Wenn}{\(R_{\mathrm{key}} = R_{\mathrm{car}}\)}
            \begin{callself}{car}{öffne Türen}{}
            \end{callself}
        \end{sdblock}
    \end{sequencediagram}
    \caption{Nachrichtenablaufdiagramm von \foreignlanguage{english}{Challenge \& Response} Verfahren.\label{fig:challenge_response_msc}}
\end{figure}

\subsection{\foreignlanguage{english}{Distance Bounding}}

In~\cite{Alrabady2005} werden verschiedene Angriffsmodelle auf das \foreignlanguage{english}{Challenge \& Response} Verfahren vorgestellt. Aufgrund der fehlenden Nutzerinteraktion in \gls{pke} sind besonders \foreignlanguage{english}{Relay}-Angriffe ein Problem. Dabei wird die eigentlich bewusst auf wenige Meter eingeschränkte Reichweite des Funkkanals durch einen Angreifer überbrückt. Der von~\citeauthor{Alrabady2005} auch als \emph{\foreignlanguage{english}{Two Thieves Attack}} bezeichnete Angriff besteht darin, dass einer der Angreifer die Funksignale am Fahrzeug abgreift und sie an seinen Mittäter mittels anders geartetem Kommunikationskanal zuschickt. Dieser positioniert sich in der unmittelbaren Nähe des Opfers, welches einen \gls{pke} Schlüssel bei sich trägt, und gibt die vom Fahrzeug gesendeten Signale über ein eigenes Funkmodul an den Schlüssel weiter. Gleichermaßen werden auch die Funksignale des Schlüssels über die so entstehende elektronische Brücke zum Fahrzeug geleitet. Als Maßnahme gegen solche Angriffe werden sogenannte \foreignlanguage{english}{Distance Bounding} Protokolle eingesetzt. Diese dienen dazu, dem Authentifizierungsverfahren eine obere Schranke bezüglich der positionellen Ausdehnung zwischen Prüfer und Prüfling zu garantieren.

Ein übliches Mittel um \foreignlanguage{english}{Distance Bounding} umzusetzen ist über die Laufzeit der Signale. Da die Signalausbreitungsgeschwindigkeit (typischerweise die Lichtgeschwindigkeit \(c\)) in den meisten Fällen als bekannt vorausgesetzt werden kann, ist eine Distanzabschätzung über die Latenz \(\tau_{l}\) und Rechenzeit der \foreignlanguage{english}{Response} \(\tau_{d}\) mittels
%
\begin{equation*}
    d = c \cdot \frac{\tau_{l} - \tau_{d}}{2} {,}
\end{equation*}
%
möglich. Beachte den Faktor \sfrac{1}{2}, da \(\tau_{l}\) üblicherweise nur als zwei-Wege Latenz messbar. Sie lässt sich unterteilen in die einfache Signallaufzeit \(\tau_{p}\) und die Rechenzeit \(\tau_{d}\):
%
\begin{equation*}
    \tau_{l} = 2 \tau_{p} + \tau_{d}
\end{equation*}

Aufgrund der hohen Ausbreitungsgeschwindigkeit sind \foreignlanguage{english}{Distance Bounding} Protokolle sehr Empfindlich gegenüber der für die Generierung der \foreignlanguage{english}{Response} benötigten Rechenzeit. Es gilt, diese Rechenzeit so gering wie möglich zu halten. Daher verzichten die meisten Ansätze auf komplexe kryptografische Funktionen zum Lösen der \foreignlanguage{english}{Challenge}. Stattdessen lassen sie sich grob unterteilen in zwei Ansätze. In beiden Fällen werden die \foreignlanguage{english}{Challenge} Bits einzeln übertragen und umgehend von einem entsprechenden \foreignlanguage{english}{Response} Bit beantwortet. Der Prüfer erzwingt die Quittierung jedes \foreignlanguage{english}{Challenge} Bits bevor er das nächste aussendet. Im ersten Ansatz zur schnellen Errechnung der \foreignlanguage{english}{Response} wird jedes \foreignlanguage{english}{Challenge} Bit mit einem Bit aus einer vorab bestimmten Bit-Folge per \foreignlanguage{english}{XOR} verknüpft~\cite{Brands1994}. Der zweite Ansatz nutzt das \foreignlanguage{english}{Challenge} Bit \(C_{i}\) als einen Index um zwischen zwei möglichen \foreignlanguage{english}{Response} Bits \(R_{i}^{0}\) oder \(R_{i}^{1}\) zu entscheiden~\cite{Hancke2005}.

\Cref{fig:distance_bounding_msc} zeigt ein den Nachrichtenablauf eines \foreignlanguage{english}{Distance Bounding} Protokolls nach \citeauthor{Brands1994}~\cite{Brands1994}. Zunächst legt der Prüfling eine zufällige Bit-Folge \(M_{1, \dots, k}\) fest. Diese hält er zunächst Geheim und gibt nur die gehashte Form \(c(M_{1, \dots, k})\) zusammen mit seiner eigenen Schlüsselkennung \(\mathit{id}\) an den Prüfer weiter. Dieser formuliert daraufhin die zufälligen \foreignlanguage{english}{Challenge} Bits \(C_{1, \dots, k}\). Damit ist der Vorbereitungsteil abgeschlossen. Nun beginnt der zeitkritische Teil des Protokolls in dem der Prüfling so schnell wie möglich auf jedes \(C_{i}\) antworten muss. Zwischen Absenden von \(C_{i}\) und dem Empfang den entsprechenden \foreignlanguage{english}{Response} \(\alpha_{i}^{\mathrm{p}}\) misst der Prüfer die Antwortzeit \(\tau_{l}\). Ist \(\tau_{l} > \tau_{l, \max}\), so befindet sich der Prüfling außerhalb des zulässigen Abstands und die Prüfung ist gescheitert. Nach dem Austausch aller \(k\) \foreignlanguage{english}{Challenge \& Response} Bits verknüpft der Prüfling die \foreignlanguage{english}{Response} Bits mit den Bits der Zusage \(M_{1, \dots, k}\) zu einer \(2k\) langen Bit-Folge. Diese wird durch unter Zunahme des geteilten Geheimschlüssels \(K\) zu einer Signatur \(h(\beta_{1, \dots, 2k}^{\mathrm{p}}, K)\) transformiert und zusammen mit \(M_{1, \dots, k}\) an den Prüfer übertragen. Dieser hat nun alle benötigten Größen, um die Echtheit des Prüflings durch Nachrechnen nachzuprüfen.

\begin{figure}
    \centering
    \begin{sequencediagram}
        \tikzstyle{inststyle}+=[text depth=0.35ex, inner sep=0.25cm]
        \newthread{prover}{Prüfling}
        \tikzstyle{inststyle}+=[below right=-0.85cm and 5cm of prover]
        \newinst{verifier}{Prüfer}

        \begin{callself}{prover}{neue Zusage}{\(M_{1, \dots, k}\)}
        \end{callself}

        \postlevel{}

        \begin{messcall}{prover}{\( {\{ \mathit{id}, c(M_{1, \dots, k}) \}} \)}{verifier}
            \begin{callself}{verifier}{neue Challenge}{\(C_{1, \dots, k}\)}
            \end{callself}
            \begin{sdblock}{Schneller Bit-Austausch für \(i = 1, \dots, k\)}{}
                \begin{messcall}{verifier}{\( {\{ C_{i} \}} \)}{prover}
                    \begin{callself}{prover}{\( C_{i} \oplus M_{i} \)}{\(\alpha_{i}^{\mathrm{p}}\)}
                    \end{callself}
                    \begin{messcall}{prover}{\( {\{ \alpha_{i}^{\mathrm{p}} \}} \)}{verifier}
                    \end{messcall}
                \end{messcall}
            \end{sdblock}
        \end{messcall}

        \begin{callself}{prover}{\(\alpha_{1}^{\mathrm{p}} M_{1} \cdots \alpha_{k}^{\mathrm{p}} M_{k}\)}{\(\beta_{1, \dots, 2k}^{\mathrm{p}}\)}
        \end{callself}

        \postlevel{}

        \begin{messcall}{prover}{\( {\{ h(\beta_{1, \dots, 2k}^{\mathrm{p}}, K), M_{1, \dots, k} \}}\)}{verifier}
            \begin{callself}{verifier}{\( C_{1, \dots, k} \oplus M_{1, \dots, k} \)}{\(\alpha_{1, \dots, k}^{\mathrm{v}}\)}
            \end{callself}
            \postlevel{}
            \begin{callself}{verifier}{\(\alpha_{1}^{\mathrm{v}} M_{1} \cdots \alpha_{k}^{\mathrm{v}} M_{k}\)}{\(\beta_{1, \dots, 2k}^{\mathrm{v}}\)}
            \end{callself}
            \begin{sdblock}{Wenn}{
                \shortstack{
                Zusage \(c(M_{1, \dots, k})\) stimmt, und \\
                \( \beta_{1, \dots, 2k}^{\mathrm{p}} = \beta_{1, \dots, 2k}^{\mathrm{v}} \), und \\
                \( h(\beta_{1, \dots, 2k}^{\mathrm{p}}, K) = h(\beta_{1, \dots, 2k}^{\mathrm{v}}, K) \)
                }
                }
                \postlevel{}
                \postlevel{}
                \begin{callself}{verifier}{Prüfung bestanden}{}
                \end{callself}
            \end{sdblock}
        \end{messcall}
    \end{sequencediagram}
    \caption{\foreignlanguage{english}{Distance Bounding} Protokoll nach \citeauthor{Brands1994}.\label{fig:distance_bounding_msc}}
\end{figure}

Andere Verfahren setzen auf Positionsinformationen aus Nebenkanälen~\cite{Wang2019}. Diese kann bspw.\@ durch \gls{gps} ermittelt werden. In der Praxis sind diese jedoch stark von den Empfangsbedingungen am jeweiligen Standort abhängig. So ist es beim Parken in der Tiefgarage aufgrund des schlechten Satellitenempfangs unwahrscheinlich, dass das Fahrzeug eine valide \gls{gps} Position ermitteln kann. Solche Verfahren werden daher hier nicht weiter erläutert.

\section{Bedrohungen}
