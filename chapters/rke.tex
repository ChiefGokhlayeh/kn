% !TeX root = ../main.tex
\chapter{\glsentrylong{rke}}

Dieses Kapitel behandelt die Funktionsweise und potenzielle Angriffe auf \gls{rke} Systeme. Zunächst wird eine kurze Historie zu verschiedenen Funktionsansätzen gegeben, um anschließend näher auf das in aktuellen Fahrzeugen weit verbreitete Hitag2 einzugehen. Schließlich wird zu jedem erläuterten Verfahren eine Reihe von Angriffen vorgestellt.

\section{Funktionsprinzip}

Bei \gls{rke} Systemen sendet der Schlüssel bei Betätigung eines Tasters auf dem Gehäuse ein kodiertes Funksignal an das Fahrzeug. Die Reichweite beträgt je nach Modell einige \SIrange{10}{100}{\meter}. Für die Funkübertragung kommen für gewöhnlich die lizenzfreien \gls{ism} Bänder infrage (EU\@: \SIlist{433;868}{\mega\hertz}, USA\@: \SI{315}{\mega\hertz}).

Die einfachsten Varianten von \gls{rke} setzen auf unidirektionale Kommunikation mittels statischer Codes (sog.\@ \foreignlanguage{english}{Fixed Codes}). \Cref{fig:fixed_code_msc} zeigt das Nachrichtenablaufdiagramm dieses Verfahrens. Unter Verzicht auf jegliche kryptografische Sicherheit, sendet der Schlüssel bei Aktivierung hierbei eine feste Bit-Folge, die einer Funktion \(f \in {\{\mathrm{open}, \mathrm{close}\}}\) und Schlüsselkennung \(\mathit{id}\) zugehörig ist~\cite{Alrabady2005}. In~\cite{Garcia2016} konnten solche Systeme bei Modellen aus dem Baujahr 2000 ausfindig gemacht werden.

\begin{figure}
    \centering
    \begin{sequencediagram}
        \tikzstyle{inststyle}+=[text depth=0.35ex, inner sep=0.25cm]
        \newthread{user}{Nutzer}
        \newinst{key}{Schlüssel}
        \newinst{car}{Fahrzeug}

        \begin{call}{user}{betätige}{key}{}
            \begin{messcall}{key}{\({\{\mathit{id}, f\}}\)}{car}
                \begin{callself}{car}{prüfe Code}{}
                \end{callself}
                \begin{sdblock}{Wenn}{Code Korrekt?}
                    \begin{callself}{car}{öffne Türen}{}
                    \end{callself}
                \end{sdblock}
            \end{messcall}
        \end{call}
    \end{sequencediagram}
    \caption{Nachrichtenablaufdiagramm von \foreignlanguage{english}{Fixed Code} \glsentryshort{rke}.\label{fig:fixed_code_msc}}
\end{figure}

\begin{figure}
    \centering
    \begin{tikzpicture}[
            block/.style={
                    draw,
                    thick,
                    text depth=0.35ex,
                    inner sep=0.25cm,
                    font=\scriptsize,
                },
            extern label/.style={
                    font=\tiny,
                },
            node distance=0.4cm,
        ]
        \node [block] (rke_rf) {RKE RF};
        \node [bareantenna, left=of rke_rf.north west, yshift=0.2cm] (rke_antenna) {};
        \node [block, below=of rke_rf] (uc) {µC};
        \node [draw=gray, very thick, fill=black, circle, right=of uc] (button) {};
        \node [draw, fit=(button)] (button_case) {};
        \node [extern label, right=0 of button_case] (button_label) {Knopf};
        \node [block, below=of uc] (immo) {Wgfsp. RFID};
        \node [bareantenna, left=of immo.north west, yshift=0.2cm] (immo_antenna) {};

        \draw (immo_antenna) |- (immo);
        \draw [dotted] (immo) -- (uc) node [extern label, anchor=west, midway] {optional};
        \draw (button_case) -- (uc);
        \draw (rke_rf) -- (uc);
        \draw (rke_antenna) |- (rke_rf);

        \draw [decorate, decoration={expanding waves, angle=20, segment length=0.15cm}] ($(immo_antenna) - (0.35, 1)$) -- ($(immo_antenna) - (0.35, 2)$);
        \draw [decorate, decoration={expanding waves, angle=20, segment length=0.15cm}] ($(immo_antenna) - (-0.35, 2)$) -- ($(immo_antenna) - (-0.35, 1)$);
        \node [extern label, below=2 of immo_antenna] {Fahrzeug Wgfsp. (\SI{125}{\kilo\hertz})};

        \draw [decorate, decoration={expanding waves, angle=20, segment length=0.15cm}] ($(rke_antenna) + (0, 1)$) -- ($(rke_antenna) + (0, 2)$);
        \node [extern label, above=2 of rke_antenna.center] {Fahrzeug RKE (\SIlist[list-separator=/, list-final-separator=/]{433;315;868}{\mega\hertz})};

        \begin{scope}[on background layer]
            \node [fill=black!5, rounded corners=0.75cm, fit=(rke_rf)(rke_antenna)(uc)(button)(button_case)(button_label)(immo)(immo_antenna)] (key_housing) {};
            \path [fill=black!5] ($(key_housing.east) + (-0.1, 0.5)$) -- ++(5, 0) -- ++(0, -0.9) -- ++(-5, 0);
            \path [fill=black!12] ($(key_housing.east) + (4.9, 0.4)$) -- ++(-0.4, 0) -- ++(-0.2, -0.2) -- ++(-0.4, 0) -- ++(-0.1, 0.1) -- ++(-0.4, 0) -- ++(-0.1, 0.1) -- ++(-0.4, 0) -- ++(-0.2, -0.2) -- ++(-0.4, 0) -- ++(-0.2, 0.2) -- ++(-0.4, 0) -- ++(-0.2, -0.2) -- ++(-0.4, 0) -- ++(-0.2, 0.2) -- ++(-0.8, 0) -- ++(0, -0.45) -- ++(0.25, -0.25) -- ++(4.55, 0) node [midway, anchor=north, extern label] {mechanischer Schlüssel};
        \end{scope}
    \end{tikzpicture}
    \caption[Aufbau eines typischen \foreignlanguage{english}{rolling-code} \glsentryshort{rke} Schlüssels mit unidirektionaler Kommunikation.]{Aufbau eines typischen \foreignlanguage{english}{rolling-code} \glsentryshort{rke} Schlüssels mit unidirektionaler Kommunikation~\cite{Garcia2016}.}
\end{figure}

\section{Bedrohungen}
