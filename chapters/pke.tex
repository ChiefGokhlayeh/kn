% !TeX root = ../main.tex
\chapter{\glsentrylong{pke}}

Gerade moderne Luxusfahrzeuge werben mit dem zusätzlichen Komfort von \glsreset{pke}\gls{pke}. Oft wird dies auch auf die elektronische Wegfahrsperre ausgeweitet und kann dann als \glsreset{pkes}\gls{pkes} bezeichnet werden.

In diesem Kapitel wird die Funktionsweise solcher Systeme beschrieben. Zunächst soll grundlegende auf das in der Praxis weit verbreitete Verfahren mittels \foreignlanguage{english}{Challenge \& Response Handshake} eingegangen werden. Dabei wird explizit die Anfälligkeit für \foreignlanguage{english}{Replay}-Attacken herausgearbeitet. Es wird außerdem ein in der Literatur seit längerem bekanntes Verfahren zur Absicherung gegen solche Angriffe vorgestellt. Die hergeleitete Theorie mit den Bedrohungen der realen Praxis verglichen. Als Beispiel dient hier das DST40 Verfahren, entwickelt von Texas Instruments und unter anderem verwendet in Luxusmodellen wie dem Tesla Model S~\cite{Wouters2019}.

\section{Funktionsprinzip}

In einem \gls{pke} System geschieht die Kommunikation mit dem Fahrzeug \emph{passiv}, d.\,h., alleinig \emph{die Präsenz} eines Schlüssels in der unmittelbaren Umgebung zum Fahrzeug reicht aus, um dieses zu entsperren. Anders als bei herkömmlichen \gls{rke} Systemen ist jenseits des Tragens eines Schlüssels keine weitere Interaktion des Nutzers notwendig. In der Praxis wird dazu i.\,d.\,R.\@ ein bidirektionaler Kommunikationskanal zwischen Fahrzeug und Schlüssel aufgebaut. Um die Batterielebenszeit des Schlüssels zu verlängern, wird die Funkkommunikation vom Fahrzeug aus initiiert. Der Schlüssel horcht dabei zunächst auf die Präsenz eines \gls{lw}[n] Signals vom Fahrzeug bevor er selbst aktiv wird. Detektiert ein Schlüssel das Signal eines Fahrzeugs antwortet er über einen \gls{ukw}[n] Funkkanal~\cite{Alrabady2005,Verdult2015,Garcia2016}.

\subsection{\foreignlanguage{english}{Challenge \& Response}}

Im \foreignlanguage{english}{Challenge \& Response} Verfahren, gezeigt in \cref{fig:challenge_response_msc}, authentifiziert sich ein Schlüssel gegenüber einem Fahrzeug, indem es eine vom Fahrzeug vorgegebene \emph{Challenge} \(C\) (engl.\@ Herausforderung) beantwortet. Gemäß Kerckhoffs's Prinzip ist der Algorithmus \(h(C, K)\) zur Lösung der \foreignlanguage{english}{Challenge} auch einem potenziellen Angreifer bekannt. Daher verwalten beide Parteien einen gemeinsamen Geheimschlüssel \(K\), der in die Berechnung der \foreignlanguage{english}{Response} \(R_{\mathrm{car}|\mathrm{key}}\) (engl.\@ Antwort) einfließt. Die Berechnung von \(h(C, K)\) wird sowohl vom Schlüssel (dem \emph{Prüfling}) als auch dem Fahrzeug (dem \emph{Prüfer}) durchgeführt. Nur wenn beide Ergebnisse übereinstimmen, gilt die Prüfung als erfolgreich. Da in die \foreignlanguage{english}{Response} normalerweise keinerlei Zusatzinformationen kodiert sind, reicht es meistens aus, für \(h(C, K)\) eine kryptografische Einwegfunktion einzusetzen. Dennoch sind auch vermehrt kryptografische Block-Chiffren in der Praxis anzutreffen~\cite{Wouters2019}.

Da der Schlüssel bedingt durch dessen Formfaktor meist über eine limitierte Batterielaufzeit verfügt, befindet sich dieser üblicherweise in einem energiesparenden Schlafmodus. Um einen potenziell in der Umgebung befindlichen Schlüssel zu wecken, sendet das Fahrzeug periodisch Weckrufe. Sobald ein Weckruf empfangen wird, antwortet der Schlüssel mit seiner Schlüsselkennung \(\mathit{id}\). Wenn diese als zulässige Schlüsselkennung im Fahrzeug hinterlegt ist, initiiert das Fahrzeug eine Prüfung des Schlüssels mittels \foreignlanguage{english}{Challenge \& Response} Verfahren.

\begin{figure}
    \centering
    \begin{sequencediagram}
        \tikzstyle{inststyle}+=[text depth=0.35ex, inner sep=0.25cm]
        \newthread{car}{Fahrzeug}
        \tikzstyle{inststyle}+=[below right=-0.85cm and 5cm of car]
        \newinst{key}{Schlüssel}

        \mess{car}{Hallo?}{key}
        \postlevel{}
        \mess{car}{Hallo?}{key}
        \postlevel{}
        \begin{messcall}{car}{Hallo?}{key}
            \begin{messcall}{key}{\({\{\mathit{id}\}}\)}{car}
            \end{messcall}
        \end{messcall}

        \begin{callself}{car}{neue \foreignlanguage{english}{Challenge}}{\(C\)}
        \end{callself}

        \begin{messcall}{car}{\({\{C\}}\)}{key}
            \begin{callself}{key}{\foreignlanguage{english}{Response} \(h(C, K)\)}{\(R_{\mathrm{key}}\)}
            \end{callself}
            \begin{messcall}{key}{\({\{R_{\mathrm{key}}\}}\)}{car}
            \end{messcall}
        \end{messcall}
        \prelevel{}
        \prelevel{}
        \prelevel{}
        \prelevel{}
        \prelevel{}
        \begin{callself}{car}{\foreignlanguage{english}{Response} \(h(C, K)\)}{\(R_{\mathrm{car}}\)}
        \end{callself}

        \postlevel{}
        \postlevel{}
        \postlevel{}
        \begin{sdblock}{Wenn}{\(R_{\mathrm{key}} = R_{\mathrm{car}}\)}
            \begin{callself}{car}{öffne Türen}{}
            \end{callself}
        \end{sdblock}
    \end{sequencediagram}
    \caption{Nachrichtenablaufdiagramm von \foreignlanguage{english}{Challenge \& Response} Verfahren.\label{fig:challenge_response_msc}}
\end{figure}

\subsection{\foreignlanguage{english}{Distance Bounding}}

\section{Bedrohungen}
