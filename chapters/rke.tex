% !TeX root = ../main.tex
\chapter{\glsentrylong{rke}}

Dieses Kapitel behandelt die Funktionsweise und potenzielle Angriffe auf \gls{rke} Systeme. Zunächst wird eine kurze Historie zu verschiedenen Funktionsansätzen gegeben, um anschließend näher auf das in aktuellen Fahrzeugmodellen weit verbreitete Hitag2 einzugehen. Schließlich wird zu jedem erläuterten Verfahren eine Reihe von Angriffen vorgestellt.

\section{Funktionsprinzip}

Bei \gls{rke} Systemen sendet der Schlüssel bei Betätigung eines Tasters auf dem Gehäuse ein kodiertes Funksignal an das Fahrzeug. Die Reichweite beträgt je nach Modell einige \SIrange{10}{100}{\meter}. Für die Funkübertragung kommen für gewöhnlich die lizenzfreien \gls{ism} Bänder infrage (EU\@: \SIlist{433;868}{\mega\hertz}, USA\@: \SI{315}{\mega\hertz}).

Die einfachsten Varianten von \gls{rke} setzen auf unidirektionale Kommunikation mittels statischer Codes (sog.\@ \foreignlanguage{english}{Fixed Codes}). \Cref{fig:fixed_code_msc} zeigt das Nachrichtenablaufdiagramm dieses Verfahrens. Unter Verzicht auf jegliche kryptografische Sicherheit, sendet der Schlüssel bei Aktivierung hierbei eine feste Bit-Folge, die einer Funktion \(b \in {\{\mathrm{open}, \mathrm{close}\}}\) und Schlüsselkennung \(\mathit{id}\) zugehörig ist~\cite{Alrabady2005}. In~\cite{Garcia2016} konnten solche Systeme bei Modellen aus dem Baujahr 2000 ausfindig gemacht werden.

\begin{figure}
    \centering
    \begin{sequencediagram}
        \tikzstyle{inststyle}+=[text depth=0.35ex, inner sep=0.25cm]
        \newthread{user}{Nutzer}
        \newinst{key}{Schlüssel}
        \newinst{car}{Fahrzeug}

        \begin{call}{user}{betätige}{key}{}
            \begin{messcall}{key}{\({\{\mathit{id}, f\}}\)}{car}
                \begin{callself}{car}{prüfe Code}{}
                \end{callself}
                \begin{sdblock}{Wenn}{Code Korrekt?}
                    \begin{callself}{car}{öffne Türen}{}
                    \end{callself}
                \end{sdblock}
            \end{messcall}
        \end{call}
    \end{sequencediagram}
    \caption{Nachrichtenablaufdiagramm von \foreignlanguage{english}{Fixed Code} \glsentryshort{rke}.\label{fig:fixed_code_msc}}
\end{figure}

Ein neueres Verfahren, gezeigt in \cref{fig:rolling_code_msc}, setzt auf sogenannte \foreignlanguage{english}{Rolling Codes}~\cite{Alrabady2005}. Hierbei wird statt einem festen Wert, ein fortlaufender Zähler übertragen. Sowohl Sender als auch Empfänger verwalten einen solchen Zähler. Der Zählerwert sowie die gewünschte Funktion \(b\) werden in moderneren Systemen vor der Übertragung üblicherweise nach einem symmetrischen Verfahren \(m(c, b)\) verschlüsselt~\cite{Garcia2016}. Nach jeder Übertragung wird der Zähler \(c_{\mathrm{key}}\) im Sender inkrementiert, sodass idealerweise nie der gleiche Zählerwert zweimal gesendet wird. Empfängt ein Fahrzeug einen solchen Code, wird dieser zunächst mittels \(m^{-1}(c, b)\) entschlüsselt. Anschließend wird der Fahrzeug-eigene Zählerwert \(c_{\mathrm{car}}\) vom Empfangenen abgezogen. Liegt der Wert unterhalb einer Toleranzschwelle \(\delta \), ist die Authentifizierung erfolgreich. Ist der Wert darüber, verweigert das Fahrzeug den Zugang.

\begin{figure}
    \centering
    \begin{sequencediagram}
        \tikzstyle{inststyle}+=[text depth=0.35ex, inner sep=0.25cm]
        \newthread{user}{Nutzer}
        \newinst{key}{Schlüssel}
        \newinst{car}{Fahrzeug}

        \begin{call}{user}{betätige}{key}{}
            \begin{messcall}{key}{\(m(c_{\mathrm{key}}, b)\)}{car}
                \begin{callself}{car}{\(m^{-1}(c_{\mathrm{key}}, b) \mapsto (c_{\mathrm{key}}, b)\)}{}
                \end{callself}
                \begin{sdblock}{Wenn}{\(c_{\mathrm{key}} - c_{\mathrm{car}} < \delta \)}
                    \begin{callself}{car}{öffne Türen}{}
                    \end{callself}
                    \begin{callself}{car}{\(c_{\mathrm{car}} \coloneqq c_{\mathrm{key}}\)}{}
                    \end{callself}
                \end{sdblock}
            \end{messcall}
            \begin{callself}{key}{\(c_{\mathrm{key}} \coloneqq c_{\mathrm{key}} + 1\)}{}
            \end{callself}
        \end{call}
    \end{sequencediagram}
    \caption{Nachrichtenablaufdiagramm von \foreignlanguage{english}{Rolling Code} \glsentryshort{rke}.\label{fig:rolling_code_msc}}
\end{figure}

Sofern die Authentifizierung geglückt ist, synchronisieren sich beide Parteien wieder auf einen gemeinsamen Zählerwert. Der Toleranzwert ist notwendig, da ein Schlüssel eventuell auch außerhalb der Reichweite des Fahrzeuges aktiviert wird. Folglich ist es jedoch auch möglich, dass Schlüssel und Fahrzeug bei wiederholter Aktivierung außerhalb der Funkreichweite die Synchronisierung verlieren und keine Authentifizierung mittels \gls{rke} mehr möglich ist. In solchen Fällen muss das Fahrzeug mechanisch aufgeschlossen werden. Eine (Re\discretionary{-)}{}{)}Synchronisierung findet über einen Seitenkanal statt. Hierzu wird in einigen Fahrzeugen der Kommunikationskanal der elektronischen \gls{wgfsp} bedient~\cite{Garcia2016}. Diese kommuniziert mit dem Schlüssel beim Einstecken in das Zündschloss über einen niederfrequenten Funkkanal. Solche Schlüssel enthalten also mitunter zwei Funkmodule, gezeigt in \cref{fig:rolling_code_key_components}. Über das \gls{rke} Modul wird der Zugang zum Fahrzeug gesteuert, während das \gls{rfid} Modul zur Kommunikation mit der elektronischen Wegfahrsperre sowie der Zählersynchronisierung dient. % chktex 9 chktex 10

\begin{figure}
    \centering
    \begin{tikzpicture}[
            block/.style={
                    draw,
                    thick,
                    text depth=0.35ex,
                    inner sep=0.25cm,
                    font=\scriptsize,
                },
            extern label/.style={
                    font=\tiny,
                },
            node distance=0.4cm,
        ]
        \node [block] (rke_rf) {RKE RF};
        \node [bareantenna, left=of rke_rf.north west, yshift=0.2cm] (rke_antenna) {};
        \node [block, below=of rke_rf] (uc) {µC};
        \node [draw=gray, very thick, fill=black, circle, right=of uc] (button) {};
        \node [draw, fit=(button)] (button_case) {};
        \node [extern label, right=0 of button_case] (button_label) {Knopf};
        \node [block, below=of uc] (immo) {\glsentryshort{wgfsp} \glsentryshort{rfid}};
        \node [bareantenna, left=of immo.north west, yshift=0.2cm] (immo_antenna) {};

        \draw (immo_antenna) |- (immo);
        \draw [dotted] (immo) -- (uc) node [extern label, anchor=west, midway] {optional};
        \draw (button_case) -- (uc);
        \draw (rke_rf) -- (uc);
        \draw (rke_antenna) |- (rke_rf);

        \draw [decorate, decoration={expanding waves, angle=20, segment length=0.15cm}] ($(immo_antenna) - (0.35, 1)$) -- ($(immo_antenna) - (0.35, 2)$);
        \draw [decorate, decoration={expanding waves, angle=20, segment length=0.15cm}] ($(immo_antenna) - (-0.35, 2)$) -- ($(immo_antenna) - (-0.35, 1)$);
        \node [extern label, below=2 of immo_antenna] {Fahrzeug \glsentryshort{wgfsp}. (\SI{125}{\kilo\hertz})};

        \draw [decorate, decoration={expanding waves, angle=20, segment length=0.15cm}] ($(rke_antenna) + (0, 1)$) -- ($(rke_antenna) + (0, 2)$);
        \node [extern label, above=2 of rke_antenna.center] {Fahrzeug \glsentryshort{rke} (\SIlist[list-separator=/, list-final-separator=/]{433;315;868}{\mega\hertz})};

        \begin{scope}[on background layer]
            \node [fill=black!5, rounded corners=0.75cm, fit=(rke_rf)(rke_antenna)(uc)(button)(button_case)(button_label)(immo)(immo_antenna)] (key_housing) {};
            \path [fill=black!5] ($(key_housing.east) + (-0.1, 0.5)$) -- ++(5, 0) -- ++(0, -0.9) -- ++(-5, 0);
            \path [fill=black!12] ($(key_housing.east) + (4.9, 0.4)$) -- ++(-0.4, 0) -- ++(-0.2, -0.2) -- ++(-0.4, 0) -- ++(-0.1, 0.1) -- ++(-0.4, 0) -- ++(-0.1, 0.1) -- ++(-0.4, 0) -- ++(-0.2, -0.2) -- ++(-0.4, 0) -- ++(-0.2, 0.2) -- ++(-0.4, 0) -- ++(-0.2, -0.2) -- ++(-0.4, 0) -- ++(-0.2, 0.2) -- ++(-0.8, 0) -- ++(0, -0.45) -- ++(0.25, -0.25) -- ++(4.55, 0) node [midway, anchor=north, extern label] {mechanischer Schlüssel};
        \end{scope}
    \end{tikzpicture}
    \caption[Aufbau eines typischen \foreignlanguage{english}{Rolling Code} \glsentryshort{rke} Schlüssels mit zwei Kommunikationskanälen.]{Aufbau eines typischen \foreignlanguage{english}{Rolling Code} \glsentryshort{rke} Schlüssels mit zwei Kommunikationskanälen~\cite{Garcia2016}.\label{fig:rolling_code_key_components}}
\end{figure}

Ein Beispiel für kommerzielle \gls{rke} Implementierungen ist das Hitag2 \foreignlanguage{english}{Rolling Code} Protokoll des Herstellers NXP Semiconductors. In~\cite{Garcia2016} wird eine tiefgehende Beschreibung des Protokolls angeführt, welche auf Basis von Rückwärtsentwicklung des Verfahrens erlangt wurde. Nach Angaben der Autoren wird dieses Protokoll von verschiedenen europäischen und US-amerikanischen Herstellern eingesetzt. Das Verfahren besteht aus einer Stream-Chiffre mit \SI{48}{\bit} Schlüssellänge~\cite{Verdult2015}. Es kommt sowohl in der elektronischen Wegfahrsperre als auch \gls{rke} zum Einsatz.

Wie in~\cite{Verdult2015} beschrieben basiert die Chiffre auf einem \SI{48}{\bit} breiten \gls{lfsr} und einem nicht-linearen Filter \(f\). Mit jedem Takt werden \SI{20}{\bit} des \gls{lfsr} durch den Filter \(f\) geschickt. Dieser bildet die \SI{20}{\bit} auf ein einzelnes Bit der Chiffrefolge ab. Anschließend schiebt das \gls{lfsr} seinen Inhalt um \SI{1}{\bit} nach links und erzeugt mittels des Generatorpolynoms die frei werdende Bit-Stelle rechts. Das Generatorpolynom ist gegeben durch
%
\begin{equation*}
    \setlength\arraycolsep{1pt}
    \begin{array}{c r r l}
        G(x) & = \left( \right. &   & x^{47} + x^{46} + x^{43} + x^{42}                      \\  % chktex 9
             &                  & + & x^{41} + x^{30} + x^{26} + x^{23}                      \\
             &                  & + & x^{22} + x^{16} + x^{8} + x^{7}                        \\
             &                  & + & {\left. x^{6} + x^{3} + x^{2} + x \right)} \mod{2} {.} % chktex 9
    \end{array}
\end{equation*}
%
Oder in alternativer Schreibweise, als Feedback-Funktion der Bit-Folge \(x_{n} \in {\{0, 1\}}\):
%
\begin{equation*}
    \setlength\arraycolsep{1pt}
    \begin{array}{r r l}
        L(x_{0}, \dots, x_{47}) = &        & x_{0} \oplus x_{2} \oplus x_{3} \oplus x_{6}         \\
                                  & \oplus & x_{7} \oplus x_{8} \oplus x_{16} \oplus x_{22}       \\
                                  & \oplus & x_{23} \oplus x_{26} \oplus x_{30} \oplus x_{41}     \\
                                  & \oplus & x_{42} \oplus x_{43} \oplus x_{46} \oplus x_{47} {,}
    \end{array}
\end{equation*}
%
mit \(\oplus \) definiert als dem \foreignlanguage{english}{XOR} Operator. Die Filterfunktion \(f\) ist unterteilt in drei Teilfunktionen \(f_{a}\), \(f_{b}\) und \(f_{c}\). Jede Teilfunktion bildet \SI{4}{\bit} auf \SI{1}{\bit} ab, wobei \(f_{a}\) und \(f_{b}\) mehrfach verwendet werden, um insgesamt \SI{20}{\bit} des \gls{lfsr} auf \SI{1}{\bit} der Chiffrefolge abzubilden. Die Filterfunktionen lauten:
%
\begin{equation*}
    \setlength\arraycolsep{1pt}
    \begin{array}{c r l}
        f(x_{0}, \dots, x_{47}) & =  f_{c} \left( \right. & f_{a}(x_{2} x_{3} x_{5} x_{6}), f_{b}(x_{8} x_{12} x_{14} x_{15}),      \\ % chktex 9
                                &                         & f_{b}(x_{17} x_{21} x_{23} x_{26}), f_{b}(x_{28} x_{29} x_{31} x_{33}), \\
                                &                         & \left. f_{a}(x_{34} x_{43} x_{44} x_{46}) \right) {,}
    \end{array}
\end{equation*}
%
mit
%
\begin{align*}
    f_{a}(i) & = {(\mathtt{0xA63C})}_{i}         \\
    f_{b}(i) & = {(\mathtt{0xA770})}_{i}         \\
    f_{c}(i) & = {(\mathtt{0xD949CBB0})}_{i} {,}
\end{align*}
%
wobei \({(a)}_{i}\) die \(i\)-te Bit-Stelle von \(a\), beginnend mit dem \gls{msb}, und \(x y\) die Verkettung der Bits \(x\) und \(y\) bezeichnet. Dies entspricht der Notation aus~\cite{Garcia2016,Verdult2015}.

\section{Bedrohungen}

\Gls{rke} Systeme mit \foreignlanguage{english}{Fixed Code} Verfahren können sehr einfach durch \foreignlanguage{english}{Replay}-Attacken umgangen werden. Wenn die \(\mathit{id}\) des Schlüssels bekannt ist (z.\,B. durch Abhorchen des Funkkanals), kann ein Angreifer beliebige gefälschte Nachrichten in den Kommunikationskanal einschleusen. Auch eine symmetrische Verschlüsselung von \( \{ \mathit{id}, b \} \) lässt sich aufgrund der Zeitinvarianz einfach umgehen, in dem der Angreifer einfach die zuvor aufgezeichneten Chiffretexte abspielt.

\foreignlanguage{english}{Rolling Codes} hingegen bieten allgemein etwas effektiveren Schutz vor unerlaubter Authentifizierung~\cite{Garcia2016}. Die Transmutation des Chiffrats durch den fortlaufenden Zähler und die Zurückweisung alter Zählerwerte im Fahrzeug können allgemein als Schutz vor \foreignlanguage{english}{Replay}-Attacken eingesetzt werden. Jedoch wird am Beispiel von Hitag2 anschaulich, dass die größte Angriffsfläche durch schwache kryptografische Implementierungen des genannten \foreignlanguage{english}{Rolling Code} Schemas bereitet wird. So konnte bereits 2012 ein kryptografischer Angriff auf die nur \SI{48}{\bit} langen Geheimschlüssel demonstriert werden~\cite{Verdult2012,Verdult2015}. Ein Korrelationsangriff auf Hitag2, gezeigt in~\cite{Garcia2016}, erlaubt dem Angreifer die Schlüsselübernahme in nur wenigen mitgeschnittenen echten Nachrichten. Unabhängig der kryptografischen Schwäche der verwendeten Chiffre zeigt~\cite{Garcia2016} auch eindrucksvoll, dass auch eine mangelhafte Geräteprovisionierung seitens der Fahrzeughersteller zu Schwachstellen im System führen kann. In einem Angriff auf ein Volkswagen-eigenes Verfahren zeigen die Autoren, dass viele Modelle der Volkswagen Gruppe mit dem gleichen Master-Key ausgeliefert werden.
